% This file is to be input where your front cover is to be; that's likely to be the beginning of your environment document.
% See the beginning of file cover_preamble.tex for explanations and instructions.

\begin{cover}
  % Draw background image
  \begin{tikzpicture}[overlay,remember picture]
    \node[coordinate] at ([yshift=-.13535\paperheight] current page.west) (P1) {P1};
    \node[coordinate] at ([xshift=\paperwidth] P1) (P2) {P2};
    \node[coordinate] at ([yshift=0.13645\paperheight] P2) (P3) {P3};
    \node[coordinate] at ([shift={(-0.33476\paperwidth,0.0669\paperheight)}] P3) (P4) {P4};
    \node[coordinate] at ([yshift=0.07173\paperheight] P4) (P5) {P5};
    \node[coordinate] at ([yshift=0.40976\paperheight] P1) (P6) {P6};
    \node[coordinate] at ([yshift=0.1346\paperheight] P3) (P7) {P7};
    \node[coordinate] at ([yshift=0.06734\paperheight] P5) (P8) {P8};
    \node[coordinate] at ([xshift=-0.33611\paperwidth] P8) (P9) {P9};
    \fill[DSColour!50,path fading=north] (P9) rectangle (P3);
    \begin{scope}
      \path[clip, scope fading=north] (P3) -- (P4) -- (P8) -- ([yshift=0.0632\paperheight] P7) -- cycle;
      \fill[pattern=customHatch,hatchcolor=hatchColourFront] (P3) -- (P4) -- (P7) -- cycle;
      \path (P8) -- node[coordinate] (P11) {P11} (P3);
      \foreach \p in {1,...,800}{
        \fill[dotColourFront] ([shift={(rand*0.17\paperwidth,rand*0.11\paperheight)}] P11) circle (\dotRadius);
      }
    \end{scope}
    \begin{scope}
      \path[clip] (P5) -- (P9) -- (P8) -- cycle;
      \fill[pattern=customHatch,hatchcolor=hatchColourFront] (P5) -- (P9) -- (P8) -- cycle;
      \path (P9) -- node[coordinate] (P10) {P10} (P5);
      \foreach \p in {1,...,400}{
        \fill[dotColourFront] ([shift={(rand*.17\paperwidth,rand*.034\paperheight)}] P10) circle (\dotRadius);
      }
      \path[fill=white, path fading = south] (P5) rectangle (P9);
    \end{scope}
    \fill[DSColour] (P1) -- (P2) -- (P3) -- (P4) -- (P5) -- (P6) -- cycle;
  \end{tikzpicture}

  % LEGACY: alternative version with image files
  % Comment out previous tikzpicture environment if you use that version
  % \begin{tikzpicture}[remember picture,overlay]
  %   \node[anchor=west,inner sep=0, outer sep=0] 
  %   at ([shift={(0mm,20mm)}] current page.west) (background)
  %   {\pgfuseimage{FrontBG}};
  % \end{tikzpicture}

  % # Typeset text
  % USER_CONFIG: it's up to you to change alignment, spacing and text formatting below
  % You may have to adjust spacing to position text with respect to the background drawing.
  \vspace{4.5cm}
  \setlength{\parindent}{0cm}% no paragraph indentation on this page
  % Introduce doctoral school
  {\scshape
    {\Huge Thèse de doctorat de}
    \vspace{2.5cm}

    \LARGE\theInstitution
    \smallskip

    comue Université Bretagne Loire
  }

  \vspace{1cm}

  {\Large
    \textsc{\'Ecole Doctorale N\textordmasculine\:\DSNumber}
    \smallskip
    
    \textit{\DSFullName}
    \smallskip

    Spécialité: \MySpeciality
  }

  \vspace{1cm}

  % Introduce thesis
  \begin{minipage}{.8\pagewidth}
    % adjust minipage width to control placing of centered and right-aligned text
    \centering
    \huge\MyTitleFRA
    \medskip

    \Large\MySubTitleFRA
  \end{minipage}

  \bigskip

  \begin{flushright}
    \huge\MyAuthor
  \end{flushright}

  \vspace{8mm}

  % Last details about thesis defense
  Thèse N\textordmasculine\:\MyThesisNumber{}, présentée et soutenue à \MyPlace, le \MyDate
  \smallskip

  Unité de recherche: \MyInstitute

  \vspace{1cm}

  % # Jury
  % USER_CONFIG: change roles of members of your jury and supervision; adapt the table parameters depending on the content.
  % - change values of \myLargeInterRow and \myInterRow to control inter-row vertical spacing
  % - change the value of \hskip in the tabular definition to control the horizontal spacing between column 2 and 3
  % - replace column type l with m or p, with width specification; e.g.: \begin{tabular}{lp{4cm}@{\hskip 1.5cm}lp{4cm}}
  % - use possibly \multirow{<n>}*{<content>} to place <content> on <n> rows if content is repeated
  \newlength{\myLargeInterRow}
  \setlength{\myLargeInterRow}{2ex}
  \newlength{\myInterRow}
  \setlength{\myInterRow}{1ex}
  \centering
  % USER_INPUT: Fill in you jury and supervision composition in the table below
  \begin{tabular}{ll@{\hskip 1.5cm}ll}
    \multicolumn{2}{c}{\textbf{Encadrement de thèse}} & \multicolumn{2}{c}{\textbf{Composition du jury}}                         \\[\myLargeInterRow]
    \multicolumn{2}{c}{Directeur de thèse}            & \multicolumn{2}{c}{Président}                                            \\[\myInterRow]
    [Prénom Nom] & [Fonction et établissement]        & [Prénom Nom] & [Fonction et établissement] \\[\myLargeInterRow]
    \multicolumn{2}{c}{Co-encadrants de thèse}        & \multicolumn{2}{c}{Rapporteurs}                                          \\[\myInterRow]
    [Prénom Nom] & [Fonction et établissement]        & [Prénom Nom] & [Fonction et établissement] \\{}%
    [Prénom Nom] & [Fonction et établissement]        & [Prénom Nom] & [Fonction et établissement] \\[\myLargeInterRow]
                 &                                    & \multicolumn{2}{c}{Membres du jury}        \\[\myInterRow]
                 &                                    & [Prénom Nom] & [Fonction et établissement] \\
                 &                                    & [Prénom Nom] & [Fonction et établissement] \\[\myLargeInterRow]
                 &                                    & \multicolumn{2}{c}{Invités}                \\[\myInterRow]
                 &                                    & [Prénom Nom] & [Fonction et établissement] \\
                 &                                    & [Prénom Nom] & [Fonction et établissement]
  \end{tabular}
\end{cover}