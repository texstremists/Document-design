\documentclass{beamer}
\usepackage{amsopn} % Provides \DeclareMathOperator and \operatorname. amsmath loads it

% Shape-dependent definitions
% N.b. \mathup provides shape-independent formatting,
% but is not always available.
\newcommand{\diffA}{\mathop{}\!\mathrm{d}}
\newcommand{\diffB}{\mathop{}\!\mathsf{d}}

% Shape-independent definitions
\newcommand{\diffC}{\operatorname{d}\!} % thin space follows
\DeclareMathOperator{\diffD}{d} % normal operator spacing

% Uncomment the following to set math font to serif
% \usefonttheme{serif}

\begin{document}

\begin{frame}
% Basic (bad) way
\[
  \int_{c}^{d}\frac{df(x)}{dx} dx
\]

% Thin spacing, works only in serif context
\[
  \int_{c}^{d}\frac{\diffA f(x)}{\diffA x} \diffA x
\]

% Thin spacing, works only in sans serif context
\[
  \int_{c}^{d}\frac{\diffB f(x)}{\diffB x} \diffB x
\]

% Thin spacing
\[
  \int_{c}^{d}\frac{\diffC f(x)}{\diffC x} \diffC x
\]

% Operator spacing
\[
  \int_{c}^{d}\frac{\diffD f(x)}{\diffD x} \diffD x
\]
\end{frame}

\end{document}
%%% Local Variables:
%%% mode: latex
%%% TeX-master: t
%%% End:
